\documentclass[luatex,fontsize=8pt,paper=b5,tate]{jlreq}
\usepackage{luwa-ul,KKsymbols}
\usepackage{gckanbun}


\begin{document}\Huge
\fboxsep=0pt\fboxrule=.1pt
今夫\送り{レ}江戸\振り{者}{は}、世之所\送り{ノ}\返り{レ}称\送り{スル}名都\振り{大}{だい}\振り{邑}{いふ}、冠蓋之所\返り{レ}集\送り{マル}\LineNumbering{\kakko{2}}[.5em]\dashKK{舟車之\振り{所}{ところ}\送り{ニシテ}\返り{レ}\振り{湊}{あつ}\送り{マル}、}実\送り{ニ}\振り{為}{た}\送り{ル}\返り{二}天下之大都会\返り{一}也。
而\送り{レドモ}\LineNumbering{\ichimoji{C}}\nolinebreak\underLineKK{其地之為名、訪之於古、未之聞}。
豈\送り{二}非\送り{ズ}\返り{三}古今相\送り{ヒ}去\送り{ルコト}\振り{日}{ひび}\送り{ニ}遠\送り{ク}、事之相\送り{ヒ}変\送り{ズルコト}愈多\送り{ク}、求\送り{ムルモ}\返り{二}其\送り{ノ}所\送り{ヲ}\返り{\IchiRe}欲\送り{スル}\返り{レ}聞\送り{カント}而不\送り{ルコト}可\送り{カラ}\返り{レ}得、亦\送り{タ}\振り{猶}{な}[ごと]\送り{ホ}[キヲ]\返り{二}今之於\送り{ケルガ}\返り{\JyouRe}古\送り{ニ}也。

% \makebox[8em][s]{%
%   \leavevmode\kern-0.25em\振り{春}{しゅん}\hfill%
%   \振り{眠}{みん}\hfill%
%   不\返り{レ}\hfill%
%   覚\送り{エ}\返り{レ}\hfill%
%   \振り{暁}{あかつき}\送り{ヲ}\kern-1.5em
% }

% \makebox[8em][s]{%
%   \振り{処}{しょ}\hfill%
%   \振り{処}{しょ}\hfill%
%   聞\送り{ク}\返り{二}\hfill%
%   \underLineKK{\振り{啼}{てい}\hfill%
%   \振り{鳥}{ちょう}\送り{ヲ}\返り{一}\kern-1em}%
% }

% \makebox[8em][s]{%
%   \振り{夜}{や}\hfill%
%   \振り{来}{らい}\hfill%
%   風\hfill%
%   雨\送り{ノ}\hfill%
%   声
% }

% \makebox[8em][s]{%
%   花\hfill%
%   \振り{落}{お}\送り{ツルコト}\hfill%
%   知\送り{ル}\hfill%
%   \振り{多}{た}\hfill%
%   \振り{少}{しょう}\kern-.25em
% }
\end{document}